\section*{Policy}
\subsection*{Merge Policy}
The repository on GitHub for the common server is michaelkamerath/SharedServer. This repository will be the development hub for all teams to implement the shared server. There are two branches in this repository: master and development. The master branch will contain all final code that has been tested and approved. Only Michael Kamerath and Chris Merrill have access to merge pull requests into the master branch. The development branch is for working changes. Teams are welcome to make other branches to merge into the development branch, but only pull requests from the development branch will be considered for merging into the master branch.\\
Pull requests for each component will be reviewed by the team assigned to review it. In case of a denial, all other teams will also review the code within 24 hours. Votes will be totaled either after all teams vote or after 24 hours, whichever comes sooner. A super majority (2/3) is required for code to pass and be eligible for merging.\\
After code is reviewed and approved, Chris and Michael will do a final review and merge code. They reserve the right to send code back for fixing in case of regression, failure to pass unit tests, lack of documentation, or failure to adhere to coding standards.
\subsection*{Testing}
Testing can be part of code reviews, but it is not necessary. Chris and Michael will checkout code and run tests for every pull request. Every public function is required to have an accompanying unit test. Absence of any unit test or failure to pass warrants immediate closure of a pull request. Testing procedures for more complex pieces will be defined as they become necessary.
\subsection*{Documentation}
Each document will have a directory. Inside the directory, there will be a main <filename>.tex file, a sections folder, a graphics folder, and other folders as necessary. The sections directory will contain .tex files for each section. These files should not have any preamble material; the first line should be \textbackslash section\{$<$sectionname$>$\} and it should only contain content for that section. Any graphics used need to be added to the graphics directory and included in $<$sectionname$>$.tex with a relative path in that directory (e.g. an image with path graphics/mysection/image.png requires \textbackslash includegraphics{mysection/image}) Changes will need to be made to $<$filename$>$.tex to include the section file (e.g. \textbackslash include\{sections/$<$sectionname$>$\}) and to reference necessary packages in the preamble.
\subsection*{Style Guide}
We'll be using Clang formatting. \href{http://clang.llvm.org/docs/ClangFormat.html}{http://clang.llvm.org/docs/ClangFormat.html} \\
Refer to \href{https://google.github.io/styleguide/cppguide.html}{Google's style guide: https://google.github.io/styleguide/cppguide.html}.
