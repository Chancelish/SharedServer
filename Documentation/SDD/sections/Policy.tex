\section*{Policy}
\subsection*{Merge Policy}
The repository on GitHub for the common server is michaelkamerath/SharedServer. This repository will be the common space for implemented features of the shared server. The master branch will contain all final code that has been tested and approved. Michael Kamerath and Chris Merrill are administrators for this repository. Only administrators have access to merge pull requests into the master branch. Other branches in this master repository should be used solely for administrative purposes and should only be created by administrators. Each team will need to fork this repository and working changes will need to be handled in the fork.\\
At the start of a new feature, two branches need to be created: a $<$feature$>$ branch and a $<$feature$>$\_cr branch. Developers should use the $<$feature$>$ branch for working changes and use a pull request to merge the feature into the $<$feature$>$\_cr branch. At this stage, all team members need to do a code review. If a super majority (2/3) of team members accept the code, it can be merged. If the feature implemented affects another team (e.g. code that will be used to interface with another segment of the project), the team(s) in question need to be referenced (@$<$team\_lead\_username$>$) in the pull request and involved in the code review process. When multiple teams are involved in the code review, each team is given one vote and code passes with a super majority.\\
Once code is approved and merged to the $<$feature$>$\_cr branch, The team leader needs to make a pull request into the master repository. An administrator will verify that the code in question has passed the necessary steps, test, and merge if all requirements are met. Along with code reviews and unit tests, these requirements include proper documentation and lack of any regression.\\
\subsection*{Testing}
Testing can be part of code reviews, but it is not necessary. Administrators will checkout code and run tests for every pull request to the master. Every public function is required to have an accompanying unit test. Absence of any unit test or failure to pass warrants immediate closure of a pull request. Testing procedures for more complex modules will be defined as they become necessary.\pagebreak
\subsection*{Documentation}
Each document will have a directory. The document directory will contain a $<$filename$>$.tex file, a sections folder, a graphics folder, and other folders as necessary. The sections directory will contain .tex files for each section. These files should not have any preamble material; the first line should be \textbackslash section\{$<$sectionname$>$\} and it should only contain content for that section. Any graphics used need to be contained in the graphics directory. Graphics included in $<$sectionname$>$.tex need to use a relative path from the graphics directory (e.g. an image with path graphics/mysection/image.png requires \textbackslash includegraphics{mysection/image}). Changes will need to be made to $<$filename$>$.tex to include the section file (e.g. add the line \textbackslash include\{sections/$<$sectionname$>$\}) and to reference necessary packages in the preamble.
\subsection*{Style Guide}
We'll be using Clang formatting. \href{http://clang.llvm.org/docs/ClangFormat.html}{http://clang.llvm.org/docs/ClangFormat.html} \\
Refer to \href{https://google.github.io/styleguide/cppguide.html}{Google's style guide: https://google.github.io/styleguide/cppguide.html}.
